\section{Problem Set}
1\; For two independent measurements $x= b_1$ and $x= b_2$, the best $\hat{x}$ should be some
weighted average $\hat{x}=ab_1+(1-a)b_2$. When $b_1$ and $b_2$ have mean zero and
variances $\sigma^2_1$ and $\sigma^2_2$, the variance of $\hat{x}$ will be $P=a^2\sigma^2_1+(1-a)^2\sigma^2_2$. Choose the number a that minimizes P: dP/da = 0.

Show that this a gives the weighting which we have claimed to be best, using weights
$w_1=1/\sigma_1$ and $w_2=1/\sigma_2$.

2\; After N = 4 coin flips (binomial distribution) what are the five probabilities $p_0,p_1,...,p_4$ of M = 0,1,..., 4 heads? Find the mean $\bar{M}=\Sigma Mp_M$.
Show that the variance $\sigma^2=\Sigma(M-\bar{M})^2 p_M$ agrees with N/4 = 1.

3\; (a) At the center of Figure 4.3 with N = 4 and $\sigma^2=N/4 = 1$, check that the actual
height $p_2=\frac{6}{16}$ is a little below the Gaussian $p(x)=1/\sqrt{2\pi}\sigma$.

(b) The center of the Gaussian with $\sigma=\sqrt{N}/2$ has height $\sqrt{2\pi N}$. Using Stirling's approximation to N! and (N/2)!, show that the middle binomial coefficient $p_{N/2}$
approaches that height: 
