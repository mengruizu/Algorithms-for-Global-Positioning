\section{Positioning by GPS}
GPS has revolutionized the science of positioning and Earth measurement. One part of that revolution is accuracy, another part is speed and simplicity. A third part is cost. All of these improvements are contributing to the growth of major applications. We frankly hope that our readers will develop new uses for GPS; the technology is ready, only imagination is needed. And the initiative to turn imagination into reality.

But this is a scientific book, not a brochure. We focus on one major advantage of GPS: accuracy. The inherent accuracy of a GPS receiver can be enhanced or degraded. It is enhanced by careful processing, it is degraded by accepting (instead of trying to eliminate) significant sources of error. We will describe EASY Suites to compute position.

We strongly emphasize the importance of time. In GPS, time is the fourth dimension. It is the reason we need at least four satellites, not three, to locate the receiver. The four coordinates to be computed are X, Y, Z, and c dt ---the speed of light multiplies the clock .discrepancy dt. This quantity cdt has the unit of distance. Since an ordinary receiver clock might only know the time within a few seconds, the elimination of error from c dt is not an optional improvement-it is absolutely required!

In short: The key to the accuracy of GPS is a precise knowledge of the satellite orbits and the time. On the ground, Keplerian elements are computed from the actually observed orbits. These elements are uploaded to the satellites' memories. The satellites carry atomic clocks (cesium and rubidium). They broadcast their own Keplerian elements for position computation at receivers. They also broadcast with lower accuracy (in the almanac) the Keplerian elements of other satellites. But it is the broadcast Keplerian elements that locate the satellite end of the line segment to the receiver. The problem of GPS positioning is to locate the receiver end.

One basic fact about angles. We are dealing GPS deserves attention. Its measurements yield distances and with trilateration and not triangulation. This has been desired centuries, because angles are definitely awkward. Of course lengths are nonlinear too, in the position coordinates X, Y, Z, c dt The receiver must solve nonlinear equations.

The purpose of this chapter is to explain how GPS works for positioning, and where mathematics is involved. We begin with inexpensive receivers and pseudoranges and limited accuracy. This is the<\$1, 000 receiver chapter, with accuracy measured by meters. Next will come the>\$10, 000 receiver chapter, with high quality receivers (or networks of receivers) that allow code and phase measurements and millimeter accuracy. Chapter 10 on DTFFERENCES OF ONE-WAY OBSERVATIONS will handle the major error sources and improve to an astonishing level the accuracy of a position.

We will also describe the MATLAB software that is freely available to the reader. For this introduction to GPS, the lecture by Ponsonby (1996) has been particularly helpful.


	\subsection{Clock Errors and Hyperbolas of Revolution}
	The goal is get a fix on the receiver's position. Suppose there were no from three satellites would provide a fix.clock errors (which is false). Then the distances Around each satellite, the known distance determines a sphere.The first two spheres intersect in a circle. Assuming that the three cut this circle at two satellites do not lie on a straight line, the third sphere will normally points. One point is the correct receiver position, the other point is somewhere out in space. So three satellites are sufficient if all clocks are correct and all ranges are measured precisely.
	
	In reality the receiver clock is inexpensive and inaccurate. When the clock error is d t, every range measured at that instant will be wrong by a distance c dt. We are the arrival time of a signal that contains information about its own departure measuring time. (The velocity of light is $c\approx300m/\mu sec.$.Of course we would use many more correct digits for c, which is slightly different in the ionosphere and the troposphere. These are among the errors to be modeled.) The incorrect range,which includes c dt from the unknown clock error, is called a pseudorange.
	
	From two satellites we have two pseudoranges $\rho 1$ and $\rho 2$.Their difference $d^{12}=\rho 1 - \rho 2$ has no error c dt from the receiver 's clock. The receiver must lie on a hyperbola of revolution, with the two satellites as the foci. This is the graph of all points in space whose distances from the satellites differ by $d^{12}$.

	The third pseudorange locates the receiver on another hyperbola of revolution (a hyperboloid) It intersects the first in a curve. The fourth pseudorange contributes a third independent hyperboloid, which cuts the curve (normally twice). Provided the the four satellites are not coplanar, we again get two Possible locations for the receiver: the correct fix, and a second point in space that is far from correct and readily discarded This is the geometry from the four pseudoranges $p^{k}$:
	\[ (X-X^{k})^{2}+(Y-Y^{k})^{2}+(Z-Z^{k})^{2}+(c dt)^{2}=(\rho ^{k})^{2} \]
	
	\subsection{Reference Ellipsoid and Coordinate Systems}
	The receiver must convert from X, Y, Z to a position on a standard geodetic reference system. For GPS this reference is WGS 84. The Russian system GLONASS uses now the slightly different reference PZ-90. Then the receiver employs a model of the geoid to compute geographical coordinates and height above sea level.  An ordinary receiver displays latitude and longiW de or Northing and Fasting in the UTM projection, which allows the user to find the position on a map. Not taking into account the correction from WGS 84 to the map projection may be the most error-prone of all! (Map projections would apply only for navigation.) Existing charts are unlikely to be accurate at the centimeter level. Still they are probably sufficient for the immediate purposes of typical users (to head toward their destination, to locate a landmark, to save their lives……).
	