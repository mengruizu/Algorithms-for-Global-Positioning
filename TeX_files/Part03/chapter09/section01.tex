\section[使用GPS定位]{使用GPS定位\\Positioning by GPS}
GPS对于定位和地球测量来说是一项革命性的科学技术。一方面在于其精确性,另一方面在于其快捷性和简易性。第三方面在于其廉价性。这些方面的提高导致了GPS方向的应用大量产生。我们真诚的希望我们的读者能够研发新的GPS方面的应用;GPS技术已经成熟,现在需要的是想法,并且主动将这些想法变为现实。

但是这是一本科学的书籍,不是一本宣传手册。我们专注于GPS的一个主要优势:精确度。GPS接收机自生的精确度是可以增强或减弱的。认真处理每一个流程可以提高精度,减弱是引入了显著的信号源误差。我们将使用EASY Suites描述计算的过程。

我们强烈要求重视“GPS时间”这一方面。在GPS定位中,时间是用于描述的第个四维度。这就是为什么我们需要至少四颗,而不是三颗卫星进行接收机定位的原因。这四个定位元素分别是$X$,$Y$,$Z$和$c\,dt$——光速乘以钟差$dt$。这个量$c\,dt$是距离的单位。由于普通的接收机时钟精确度仅仅只能达到秒一级,那么消除$c\,dt$的误差就不再是一个可选的,而是完全必须要处理的改进。

总而言之,GPS定位精确性的关键在于准确的卫星轨道和时间信息,在地面上,开普勒参数是通过实际观测到的轨道来计算的。这些参数上传到卫星的存储器里。卫星里携带着原子钟。它们分别播报它们自己的开普勒参数并在接收机计算。它们同时会播报低精度的其他卫星的开普勒参数。但是这些播报的参数是当前卫星轨道上最后的一部分。GPS定位的主要问题就是定位接收机。

关于GPS有一个事实我们需要注意,就是GPS测量提供的是距离而不是角度。我们通过三边测量的方法而不是三角测量的方法,这是几个世纪以来我们一直想要的,因为角度测量是绝对不方便的。当然在定位的元素$X$,$Y$,$Z$和$c\,dt$中,距离测量也是非线性的。接收机必须解决非线性的方程组。

这一章节的目的是解释GPS是如何进行定位和数学上的关系。我们将从廉价的接收机,伪距测量和有限制的精度开始。这一章使用的是小于\$1000的接收机,测量精度将达到米级。下一章我们将使用大于\$1000的接收机,高质量的接收机(或GPS网)允许使用组合码和相位测量,测量精度将达到毫米级。第十章“差分观测的方法”将处理主要的误差源并且将定位的精度等级提高到令人惊讶的程度。

同样的我们认为MATLAB软件是读者们可以自由获取的。对于这个GPS的前言,Ponsonby(1996)的讲座是特别有帮助的。
	\subsection[钟差和双曲面]{钟差和双曲面\\Clock Errors and Hyperbolas of Revolution}
	现在目标是获得一个可靠的接收机坐标。假设它们没有钟差。那么有三颗卫星到接收机的距离就可以确定一个点。每一个卫星都有一个的距离话可以在空间钟确定一个球体。首先两个球体相交可以获得一个圆。假设三个卫星不再一条直线上时,第三个球体通常和圆交于两点。一点是正确的接收机位置,另一点通常在太空中。所以三颗卫星足够定位,如果时钟是准确的并且所有的距离测量都是精确的。
	
	实际上接收机的时钟很廉价而且不精确。当钟差是$dt$时,每一个距离测量就会立即多出一项距离误差$c\,dt$。我们测量信号到达时间,其中包含的信息有发出时间。(光速大约是$c\approx300m/\mu sec.$ 当然我们使用更加精确的位数表示$c$,这对电离层误差和对流层误差有略微的不同。它们是模型误差之一。)这个不准确的测量值包含着来自未知钟差的$c\,dt$,被称为伪距。
	
	有两个卫星时我们可以获得两个伪距观测$\rho 1$和$\rho 2$。它们的差值$d^{12}=\rho 1 - \rho 2$就没有接收机误差$c\,dt$。接收机必须位于双曲线上,两个卫星分别为焦点。这个曲线图上所有的点在空间中的距离到两个卫星的差值都是$d^{12}$。
	
	第三个伪距值确定接收机在其他的双曲线上(即双曲面)。它们首先相交于曲线。第四个伪距值提供了三个独立的双曲面,它们相切在曲线上(通常是两个)。假如四个卫星不是公面的,我们再一次的获得了两个接收机可能的位置:一个准确位置、另一个在太空中,距精确值偏离了太多所以舍弃了。这个几何结构来自四个伪距观测$p^{k}$:
	\[ (X-X^{k})^{2}+(Y-Y^{k})^{2}+(Z-Z^{k})^{2}+(c dt)^{2}=(\rho ^{k})^{2} \]
	
	\subsection[参考椭球和坐标系统]{参考椭球和坐标系统\\Reference Ellipsoid and Coordinate Systems}
	接收机必须将$X$、$Y$、$Z$坐标转换至大地测量参考标准。对于GPS这个参考是WGS84。俄罗斯的GLONASS的参考系略微有所不同是PZ-90。然后接收机使用一个大地水准体模型计算地理坐标系和海拔高度。通常接收机显示纬度和经度或者在UTM投影上的北距和东距,这样可以使使用者在地图上找到坐标位置。不考虑WGS84到地图投影的修正值的话将极易导致错误。(地图投影仅仅是用来导航)实际上,地图不太可能达到厘米级的精度。尽管如此它们大概足够接近平常用户的用途了。
	\newpage