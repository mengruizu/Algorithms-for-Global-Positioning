\section[计算模型的动机]{计算模型的动机\\Motivation for Computational Models}

假设两台接收机的距离是十分接近的(这里指相距50km之内)。来自两万公里以外的卫星信号将沿着相似的路径到达两台接收机。信号在电离层的延迟是非常相似的。卫星钟差、卫星轨道误差都是相同的。那些误差都将在两台接收机的作差中消除。

时间同步、大气影响和轨道误差对于分离的站点是成正比的。电离层梯度通常为$1^6$数量级(???)。对于单频精密定位,两测站相距100km所产生的误差太大而不能忽视(误差影响可达到10mm甚至超过$1\sim2km$)。对于任何距离的两个站点利用双频观测值几乎都能消除这些误差。轨道计算误差在$1\sim2m$,这就等同于超过100km就会产生5mm的误差。这对地质构造是重要的,而对常规测量或是导航无关紧要。
差分GPS(DGPS)的思想十分简单。如果基准站的坐标位置精确已知,那么很容易计算出它与每颗卫星之间的距离,然后就可以获得以光速传播的传播时间(距离除以光速c)。通过将理论的传播时间与实际的传播时间进行对比获得每颗可视卫星的误差项,并将这些误差项传送给基准站。第二台接收机(或者任意数量的流动站)将会接收这些信息并对相同卫星的传播时间进行改正。

因此,增加额外的收发接收机能够使我们消除电离层的部分自然误差。

差分的工作原理甚至被更多的应用在GPS数据处理的过程中。在本章中将解释双差的含义并不断被应用。利用DGPS原理显著提高了定位精度。

在定位与导航中,角度是传统的观测值,它是大地测量学的基础。但自从20世纪60年代起,测量工作者在科学研究中除了使用角度观测量以外还使用电子测量的距离。对于GPS来说,距离是基本观测值。因此三边测量在很大程度上取代了三角测量。

从大地测量学的观点来看,GPS的关键特性是用户可以随时计算所有可见卫星的位置。在第9章已经描述过计算卫星位置的复杂算法。根据卫星信号携带的信息,接收机可以立即获取卫星坐标。这些坐标涉及到地心地固坐标系,接收机的地心坐标系也与此系统有关。任何好的接收机都能够在开机两分钟之内提供位置信息。

从概念上讲,卫星可以被看作是空中坐标已知的点,并不断向地面发送它们的位置信息。接收机通过测量每一颗卫星到它的距离计算得到自身坐标$(X_i,Y_i,Z_i)$,定位精度在$2\sim10m$之间变化。如果使用位置已知的固定接收机所提供的补偿值进行改正,定位精度会大幅提高。这就是差分GPS。大地测量学中用差分GPS的方式定位可以达到厘米级的精度。

频率**(行内公式P320正数第一段)**=10.23MHz是GPS的基准频率。每颗卫星都发射两种频率的载波(此为在2011年之前,大约在2013年一个新的民用频率L5将用有完全的运行能力)。L1信号的频率为**(行内公式P320正数第一段)**=154 =1575.42MHz,波长为**(行内公式P320正数第一段)**=0.1905m;L2信号的频率为**(行内公式P320正数第一段)**=120 =1227.60MHz,波长为**(行内公式P320正数第一段)**=0.2445m。这两个频率是相干的因为154和120都是整数。L1载波搭载一个精码和一个粗/捕获码,L2载波仅搭载精码P码。导航信息数据就叠加在这些码上。

利用相位观测值可以计算得到最准确的距离,相位观测值是传入的卫星信号的相位和接收机产生的相同频率信号的相位之间的差值。相位差值来自于锁相环而且分辨率能够达到**(行内公式P320正数第二段)**周或是更好。初始观测值仅由相位差值的小数部分组成。在连续跟踪卫星信号且没有发生失锁时,从起始历元开始记录相位差值的小数部分和一部分整周部分。但是相位观测值不提供初始整周数即模糊度N。

记录没有发生周跳的周数是GPS定位中求解模糊度至关重要的方法。这尤其适用于时间较短的双差观测,例如几个小时观测甚至更短。单向观测和单差观测是不能解决模糊度问题的,因为不能消除(10.2)中的**(行内公式P320正数第三段)**项。

精密GPS定位精度的高低高度依赖于估计的卫星钟差和接收机钟差的好坏。由于光速在0.01纳秒传播3mm,因此利用可用的GPS观测值达到毫米级的精度就要求我们获取的时间精度小于纳秒级。

标准的观测方法是指利用两台及以上接收机观测四颗及以上卫星获取相位观测值并按历元存储。每个历元都是一个瞬时时间。对于双差观测值我们只需要获取5微秒以内的抽样历元即可且大多数从P码伪距获取。双差观测只要求接收机振荡器具有良好的短期稳定性,而你的腕表具有更好地长期稳定性,详见表格9.5。

原始观测值是一台接收机和一颗卫星之间的差值(单向),两台接收机对同一颗卫星的估计误差是相同的。接下来我们对站际星际作差即双差,这样得到的双差估计误差对于两台接收机是相同的。最大的误差是接收机钟的偏移,而接收机的频漂是难以估计的。

差分思想对于修复两台接收机的同步误差是有效的。多路径效应是无法消除的,因为它取决于每台接收机所处的具体反射表面的特性。更合理的天线设计和信号处理是当前研究的目标。

我们可以在双差的基础上对两个历元继续作差。利用三差(即站际、星际和历元际)结果估计模糊度N。随着时间的推移,模糊度是恒定不变的。由于作差的影响三差观测值具有较差的几何强度,具有很强的相关性且数值不稳定。	

GPS卫星信号可能会由于建筑物或者构筑物的遮挡而不连续即发生周跳。当失锁的卫星信号再次建立时,整周模糊度很可能是错误的,因此需要重新固定。现如今周跳探测由非常高效的算法实现,该算法假设历元与历元之间的电离层变化是很小的。因此组合观测值**(行内公式P321正数第一段)**对周跳是非常敏感的。当时间间隔小于几秒时,电离层变化可以到达亚厘米级。由于周跳导致的距离误差是波长**(行内公式P321正数第一段)**的倍数,这对严谨的科学研究的影响是十分巨大的。

搭载在高频率L1、L2载波上的卫星信号能够相对容易的穿过电离层。电离层延迟与载波频率的平方成反比,双频接收机利用这个关系可以估计得到大部分的电离层延迟。如果我们能够正确估计出模糊度N的整周数则称之为固定解(对模糊度),而浮点解只能得到一个不准确的非整数解N。

数据处理的最后一步是利用最小二乘原理估计点的坐标,点坐标值作为估计向量。估计值取决于对观测值质量的评估和对可能总误差的检测。GPS测量结果的是通过最小二乘估计得到其位置的网络点。

例子10.1(矩阵**(行内公式P321正数第四段)**和**(行内公式P321正数第四段)**的特性)一个典型的系数矩阵**(行内公式P321正数第四段)**可以描述GPS单一定位的情况。每行包含每颗卫星位置三个方向分量的余弦。第四列是未知的接收机钟偏差,设置为1。
**P321公式**
分别计算前三列的平均值获取各平均方向
**P321公式**
对矩阵**(行内公式P321正数第二段)**前四行(四颗卫星的观测值)取逆可得4×4的块矩阵:
**P321公式**
块矩阵有一个非常明显的特性:前三行每一行的值相加为零,第四行为1。为什么会出现这种情况呢?(因为矩阵**(行内公式P321正数第二段)**的各行元素与矩阵**(行内公式P321正数第二段)**的每列元素是正交的。)

未知量的协方差矩阵由**(行内公式P321倒数第一段)**计算得到,并在4×4的协方差矩阵中选择相关的块矩阵得到三个坐标:
**P321公式**
协方差矩阵**(行内公式P322正数第一段)**的特征值为:
**P322公式**
特征向量为:
**P322公式**

有人可能认为方向向量**(行内公式P322正数第三段)**是与矩阵**(行内公式P322正数第三段)**第三列的特征向量对准的。所有的计算以下列事实为基准:所有被追踪的卫星都在当地地平线以上且其平均方向大致是向上的;提供接收机在地球半径圈上较好的初始近似位置(???)。所有的计算都在M文件的testA.m中执行。
上述我们简要的介绍了如何使用GPS实现定位的,下面我们将着重说明GPS计算方面的问题。
