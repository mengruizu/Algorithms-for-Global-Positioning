\section[最小二乘方程]{最小二乘方程\\The Equations of Least Squares}
	本章提出了全球定位的基本数学。在大图片中,我们获得卫星到接收机的伪距测量值。有如此多的卫星,我们希望有更多测量值而不是最少的测量值去确定X,Y,Z,t(位置和时间)。但是这些测量值包含误差(所以我们使用“假,伪”这一词汇去描述观测到的距离)。这就是最小二乘算法:不相容方程组最好的解决方法。
	典型的问题是超定线性系统有噪声e(误差):
	。。。。。。(4.1)
	通常我们有适合n个参数的m个方程(m为观测值)。(未知数x1,x2,.....,xn可以表示接收机的位置,速度,或时间)。方程。。。无解,因为m>n。我们的任务是找到方程最优解。。。(被称为“x帽子”)。
	本章建立方程,当噪声向量e是一个随机变量时确定x。从最简单的假设开始:m的组成部分e独立于标准正态分布(均值为0,方差为1),没有理由相信任何方程在。。。中比其他方程多或少。未加权的最小二乘最小化为。。。,即平方差之和,最佳估计。。。解答正规方程(标准式方程)
	。。。。。。(4.2)
	我们将通过微积分和几何代数得到这些方程,通过通常条件。。。微积分最小化。。。
	通过选择接近于b,组合Ax的A的列数几何最小化E。代数承认。。。是b的投影,在子空间内所有可能组合Ax。
	所有理解正规方程。。。的方式值得你的关注。也许这是最重要的线性代数中的应用。
	解决这些方程数值是一个单独的问题。在大多数情况下,我们只是从系数矩阵。。。(对称正定)入手,然后通过消除不旋转解决。虽然这种方式可以解决,却不是最佳方式:
	1. 当。。。为病态(系数矩阵A几乎是依赖的列数)时,可能需要更大的数值稳定性,然后有了避免形成。。。的“平方根法”。系数矩阵A相反的列可以提前使其正交化,这在第五章中描述。这个著名的算法使用克-施密特方法,但户主矩阵给出一个更好的方法,奇异值分解可以发挥重要作用。
	2. B中的观测值不一定立刻生成,在这种情况下,我们更喜欢递归最小二乘,更新估计的。。。来反映最新的观测结果,从一开始就没有重复之前的步骤和验算。
	在动态应用(比如移动接收机)中,其状态本身是不断变化的。我们从第i步新的观测中估计不同的的位置向量xi,但这个向量xi与之前的xi-1通过状态方程建立联系,所以每一步是一个两阶段的过程:从状态方程预测(有自己的误差来源),后跟一个修正值去说明下一步新的观测值。
	卡尔曼滤波执行这两个步骤。我们在第八章解释这一想法和推导更新公式,在4.3节举出一个例子。
	3. b中的观测值m可能不是同样地可靠(它们可能不是独立的,误差e1,...,em可能是相关的)。b中的不确定性通过方差。。。来评估,ei与ej的相关性通过协方差。。。来评估。所有这些数字放入协方差矩阵。。。中,描述如下:
	对称矩阵。。。告诉我们观测方程Ax=b正确的权重,普通的最小二乘有隐含的假设,即。。。,独立且平等的方差确定相等的权重,方差越大的方程,可靠性越低,权重越小。
	我们将展示逆矩阵。。。为何是加权最小二乘的正确选择。正规方程从。。。变为。。。,这些方程直接解或者递归解。这一重要的结果不仅是最佳估计。。。,而且还是协方差矩阵。。。,用来评定最佳估计的可靠性。
	本章解释概率分布和协方差。正态分布(或高斯分布)中的因素。。。总是占主导地位,当协方差阵。。。包含在内时系数变为。。。,这些是基本的想法,定位的应用是这本书的核心。
	例4.1 最小二乘的一个重要应用是由m个点拟合一条直线。从3个点开始:找到最接近(0,6),(1,0)和(2,0)的线。
	图4.1 最好的线和投影:两张图片,同样的问题。这条线有顶点p=(5,2,-1),误差e=(1,-2,1)。方程式。。。给出结果。。。,最好的线为5-3t,第二张图片中,b的投影为p=5a1-3a2。
	没有直线b=C+Dt通过这三个点。我们要求两个数C和D,满足三个方程,这是方程在t = 0,1,2匹配给定值b = 6,0,0:
	t=0   第一个点在直线b=C+Dt  则  C+D*0=6 
	t=1   第二个点在直线b=C+Dt  则  C+D*1=0
	t=2   第三个点在直线b=C+Dt  则  C+D*2=0。
	这个3×2的系统没有解决方案:b =(6,0,0)不是列(1,1,1)和(0,1,2)的组合。从这些方程中读取A,x和b,解答。。。:
	缺少矩阵
	Ax=b是不可解的。
