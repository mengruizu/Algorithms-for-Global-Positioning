\section[离散时间中的随机过程]{离散时间中的随机过程\\Random Processes in Discrete Time}
	第四章从连续随机变量开始,本章我们用类似的方法进行。(这儿有一段看不清)
	
		\[ x_{k}=F_{k-1}x_{k-1}+G_{k}\varepsilon_{k} \]
		\begin{equation}\label{5.24}
	b_{k}=A_{k}x_{k}+e_{k}
	\end{equation}
	
	
	假设状态方程中的不相关过程噪声 $ \varepsilon_{k} $ 有协方差矩阵 $\sum_{\varepsilon,k} $ 。假设观测噪声  $ e_{k} $ 是不相关的并且具有零均值,通过协方差传播率,我们得到关于协方差 $ x_{k} $ 的以下递归方程:
	
	\begin{equation}\label{5.25}
	\sum\nolimits_{k}=F_{k-1}\sum\nolimits_{k-1}F_{k-1}^{T}+G_{k}\sum\nolimits_{\varepsilon,k}G_{k}^{T}
	\end{equation}
		然而,模型(5.24)允许随机过程噪声 $ \varepsilon_{k} $ 的时间相关。这种相关常常发生在实践的模型中。它可以通过增加状态向量 $ x_{k} $  来正确处理。假设 $ \varepsilon_{k} $ 可以分为相关量 $ \varepsilon_{1,k} $ 和不相关量 $ \varepsilon_{2,k}: \varepsilon_{k} = \varepsilon_{1,k} + \varepsilon_{2,k} $,我们假设$\varepsilon_{1,k} $可以模型化为差分方程
		
			\[ \varepsilon_{1,k}=G_{\varepsilon,\varepsilon_{1,k}}+\varepsilon_{3,k-1} \]
			
		 $ \varepsilon_{3} $	是不相关噪声的向量,增强状态向量 $ x_{k}^{'} $ 通过以下式子给出
		 
		 \[ x_{k}^{'}=\begin{bmatrix} x_{k}  \\ \varepsilon_{1,k}\end{bmatrix} \quad \]
		 
		 增强状态方程仅由不相关干扰给出
		 
		 	\begin{equation}\label{5.26}
		 x_{k}^{'}=\begin{bmatrix} x_{k}  \\ \varepsilon_{1,k}\end{bmatrix} \quad=\begin{bmatrix} F&G  \\ 0&G_{\varepsilon}\end{bmatrix} \quad \begin{bmatrix} x_{k-1}  \\ \varepsilon_{1,k-1}\end{bmatrix} \quad
		 + \begin{bmatrix} G&0 \\ 0&I\end{bmatrix} \quad \begin{bmatrix} \varepsilon_{2,k-1}  \\ \varepsilon_{3,k-1}\end{bmatrix} 
		 \end{equation}
		 接下来我们考虑系统扰动的四个特定模型。在每种情况下以标量描述。
		 
		 \textbf{实例5.7}(随机常数)随机常数是有固定随机振幅的非动态量,该过程由以下方程描述
		 
		 \[ x_{k}=x_{k-1} \]
		 
		 随机常数可能有随机初始状态 $ x_{0} $。
		 
		 \textbf{实例5.8}(随机游走)该过程由以下方程描述
		 
		 	\[ x_{k}=x_{k-1}+\varepsilon_{k} \]
		 
		 噪声的方差为
		 
		 \[ E\left\lbrace \varepsilon_{k}^{2}\right\rbrace = E\left\lbrace(x_{k}-x_{k-1})^{2} \right\rbrace =E\left\lbrace x_{k}^{2} \right\rbrace +E\left\lbrace x_{k-1}^{2} \right\rbrace -2E\left\lbrace x_{k}x_{k-1} \right\rbrace =2\sigma_{x}^{2} \]
		 
		 \textbf{实例5.9}(随机斜坡)随机斜坡是随时间线性增长的过程,随机斜坡的增长率是具有给定方差的随机量。我们需要两个状态元素来描述随机斜坡:
		 
		 	\[ x_{1,k}=x_{1,k-1}+(t_{k}-t_{k-1})x_{2,k-1}+\varepsilon_{1,k-1} \]
		 \[ x_{2,k}=x_{2,k-1}+\varepsilon_{2,k} \]
		 
		 \textbf{实例5.10}(指数相关随机变量)
		 
		 \[ x_{k}=e^{-\alpha(t_{k}-t{k-1})}x_{k-1}+\varepsilon_{k}\]
		 
		 我们有$ \varepsilon_{k}=x_{k}-e^{-\alpha(t_{k}-t_{k-1})} $ 。时间差是 $ \Delta t=t_{k}-t_{k-1} $。根据(5.20)我们有$ E\left\lbrace \varepsilon_{k}^{2}\right\rbrace =\sigma^{2}(1-e^{-2 \alpha   \bigtriangleup t  })  $.
		 
		 实例5.7-5.10是许多线性滤波器的基础,接下来我们讨论三个和GP应用相关的例子,见 Axelrad and Brown (1996)
		 
		 \textbf{实例5.11}(离散随机斜坡)通常随机误差表现出确定的时间增长行为。离散随机斜坡是一个随时间线性增长的函数,通常可以用来描述随机误差,随机斜坡的增长率是具有给定方差的随机量。这个模型的很好的例子是偏移 $  b $ 的行为和接收机钟的漂移 $ d $ 。
		 
		 需要两个状态分量来描述随机斜坡,所以我们用向量 $ x_{k} $ 和矩阵方程:
		 
		 
		 	\begin{equation}\label{5.27}
		 x_{k}=Fx_{k-1}+\varepsilon _{k} \quad with \quad  x_{k}= \begin{bmatrix} b_{k}  \\ d_{k}\end{bmatrix} \quad and\quad F=\begin{bmatrix} 1&\Delta t  \\ 0 & 1 \end{bmatrix}
		 \end{equation}
		  
		 
		 偏移 $ b $ 是随机斜坡过程。漂移 $ d  $ 描述了斜坡的斜率。$ F $ 的第二行给了从 $ d_{k-1} $ 到 $ d_{k} $ 的斜率的变化。第一行给了随机游走:
		 
		 \[ b_{k}=b_{k-1}+\Delta t d_{k-1}+random error \]
		 
		 
		 接下来,我们估计观测误差 $ \sum = E\left\lbrace \varepsilon \varepsilon^{T} \right\rbrace  $  的协方差矩阵。我们从一个连续的系统公式开始,
		 
		 并整合一个时间步骤:
		 
		  \[ \epsilon_{k}=\int_{t_{k-1}}^{t_{k}}F(t_{k},\tau)\epsilon(\tau)d\tau \]	
		  
		  这产生
		  
		 	\begin{equation*}
		 \begin{aligned}
		 E\left\lbrace \epsilon_{k} \epsilon_{k}^{T} \right\rbrace &=E\left\lbrace\int_{t_{k-1}}^{t_{k}}\int_{t_{k-1}}^{t_{k}} F(t_{k},\tau)\epsilon(\tau)\epsilon(\sigma)^{T}F(t_{k},\sigma)d\tau d\sigma \right\rbrace \\
		 &= \int_{t_{k-1}}^{t_{k}} F(t_{k},\tau)\sum(\tau)F(t_{k},\tau)^{T}d\tau
		 \end{aligned}
		 \end{equation*}
		 
		  矩阵 $\sum (\tau)$ 是一个谱密度矩阵。让偏移和漂移的频谱幅度为  $ s_{b} $ 和 $ s_{d}$ :
		  
		  	\[ \sum = \begin{bmatrix} s_{b}&0  \\ 0&s_{d} \end{bmatrix} \quad \]
		  	
		  然后被积函数是一个2*2的矩阵
		  
		  \[ F\sum F_{T} = \begin{bmatrix} s_{b}+s_{d} \tau ^{2} & s_{d}\tau \\s_{d}\tau & s_{d} \end{bmatrix} \quad  \]
		  
		  然后我们得到了协方差矩阵的公式:
		  
		  	\begin{equation}\label{5.28}
		  \begin{aligned}
		  E\left\lbrace\epsilon \epsilon^{T} \right\rbrace &=\int_{t_{k-1}}^{t_{k}} \begin{bmatrix}
		  s_{b}+s_{d}\tau^{2}&s_{d}\tau\\s_{d}\tau&s_{d}
		  \end{bmatrix}\quad d\tau\\
		  &=\begin{bmatrix}
		  s_{b}\Delta t+s_{d}(\Delta t)^{3}/3&s_{d}(\Delta t)^{2}/2\\s_{d}(\Delta)^{2}/2&s_{d}\Delta t
		  \end{bmatrix}
		  \end{aligned}
		  \end{equation}
		  
		  白噪声的频谱幅度$ s_{b} $和接收机钟差 $ s_{d} $ 的典型值为  $ 4\times10^{-19} $ 和 $ 15\times10^{-19} $ 。
		  一个典型的时间步长是  $ \Delta t=20s $ 。在这种情况下协方差矩阵 $ \sum\nolimits_{clock} $ 是:
		  
		  	\[ \sum\nolimits_{clock}=E\left\lbrace \epsilon \epsilon^{T} \right\rbrace =\begin{bmatrix}
		  400004&300\\300&3000
		  \end{bmatrix}\times10^{-19} \] 
		  
		  \textbf{实例5.12}一个GPS接收机的过程模型包括与接收机钟偏和钟漂相组合的接收机的三个坐标。动态过程仍由(5.27)给出,状态向量X有五个分量:
		  
		  	\[ x_{k}=\begin{bmatrix}
		  x\\y\\z\\b\\d
		  \end{bmatrix}\quad and \quad  F=\begin{bmatrix}
		  1&0&0&0&0\\0&1&0&0&0\\0&0&1&0&0\\0&0&0&1&\Delta t\\0&0&0&0&1
		  \end{bmatrix}  \]
		  
		  静态接收机的协方差矩阵是:
		  
		 	\begin{equation}\label{5.29}
		 \sum\nolimits_{static} = E\left\lbrace \epsilon \epsilon^{T} \right\rbrace = \begin{bmatrix}
		 \sum\nolimits_{position}&0\\0&\sum\nolimits_{clock}
		 \end{bmatrix}
		 \end{equation}
		  
		  
		  矩阵 $ \sum\nolimits_{clock} $ 反应了接收机钟的随机分布。协方差矩阵 $ \sum\nolimits_{position} $ 反映了和测站相关的模型噪声。当接收机放在固定的位置(静态接收机)很自然的设为 $ \sum\nolimits_{position}=0$。然而这意味着所有新的测站信息将被忽略并且没有意义。所以我们人为的让测站有一个小的偏差以便滤波不会卡住。
		  
		  \textbf{实例5.13}运动接收器是一个可以四处移动的GPS接收器,他通常放在低速移动并且不会突然变速的车上。现在状态向量包含八个分量:
		  三个坐标,三个速度,和两个时钟项:
		  
		  	
		  \[ x_{k}=\begin{bmatrix}
		  x\\y\\z\\\dot{x}\\\dot{y}\\\dot{z}\\b\\d
		  \end{bmatrix}\quad 和 \quad  F=\begin{bmatrix}
		  1&0&0&\Delta t&0&0&0&0\\0&1&0&0&\Delta t&0&0&0\\0&0&1&0&0&\Delta t&0&0\\0&0&0&1&0&0&0&0\\0&0&0&0&1&0&0&0\\0&0&0&0&0&1&0&0\\0&0&0&0&0&0&1&\Delta t\\0&0&0&0&0&0&0&1
		  \end{bmatrix}  \]
		  
		  协方差为
		  
		 \begin{equation}\label{5.30}
		 \sum\nolimits_{kinematic}=E\left\lbrace \epsilon\epsilon^{T}   \right\rbrace=\begin{bmatrix}
		 \sum\nolimits_{position}&\sum\nolimits_{position,velocity}&0\\\sum\nolimits_{position,velocity}&\sum\nolimits_{velocity}&0\\0&0&\sum\nolimits_{clock}
		 \end{bmatrix} 
		 \end{equation}
		  
		  矩阵 $ \sum\nolimits_{velocity} $ 通常对水平分量和垂直分量使用不同的值,汽车基本不改变它的垂直速度,但它可以快速加速或减速。当然,如果  $ \sum\nolimits_{kinematic} $ 中的对角项的方差很大,如下一张描述的滤波过程,将不会对位置的精度有明显提高。
		  
		  
		 \textbf{ 实例5.14}(高斯-马尔克夫过程)令 $ x_{k} $ 为具有零均值并且自相关指数递减的静态随机过程:
		 
		 \[ R_{x}(t_{2}-t_{1})=\sigma^{2}e^{-\alpha|t_{2}-t_{1}|} \] 
		 
		 
		 当输入  $v\varepsilon_{k} $ 是具有等于1的功率谱密度的零平均白噪声时,这种类型的随机过程可以被建模为线性系统的输出。 (在标准时间序列文献中,这被称为AR(1)模型.AR(1)表示阶数1的自回归)。这种类型的过程的差分方程模型是:
		  \[  x_{k}=Fx_{k-1}+G\epsilon_{k}\]
		 \begin{equation}\label{5.31}
		b_{k}=x_{k}
		\end{equation}
		 
		 为了使用这个模型,我们需要求解未知标量参数 $ F $ 和 $ G $ 作为参数  $ \alpha $ 的函数。 为此,我们在两边乘以(5.31)两边并取
		\[ \textbf{Table 5.2} System models of discrete-time random processes  \]
		\[ \begin{tabular}{lcr}
		\hline
		$ Process\quad type $ & $ Autocorrelation\quad R_{x}(\tau) $ & $State\quad model $ \\
		\cline{1-3}
		$ Random \quad constant $ & $  R_{x}(\Delta t)=\sigma^{2}$ & $  x_{k}=x_{k-1},\sigma^{2}{x_{0}}=\sigma^{2}$\\
		$ Random \quad walk $&$  R_{x}(\Delta t)= \infty  $&$  x_{k}=x_{k-1}+\epsilon_{k},\sigma^{2}{x_{0}}=0$\\
		
		$ Random \quad ramp $ &$ \quad$&$  x_{1,k}=x_{1,k-1}+\Delta t x_{2,k-1}$\\
		
		$  \quad  $ &$ \quad$&$  x_{2,k}=x_{2,k-1} $\\
		
		$  Exponenitially  $  & $ R_{x}(\Delta t)=\sigma^{2}e^{-\alpha|\Delta t_{k}|} $ & $ x_{k}=e^{-\alpha|\Delta t|}x_{k-1}\epsilon_{k}$\\
		
		$ correlated$ & $  \quad   $ & $ \sigma^{2}{x_{0}}=\sigma^{2},\Delta t = t_{k}-t_{k-1} $ \\
		
		\hline
		\end{tabular} \]      
		 期望值以获得方程
		 
		 \[ E\left\lbrace x_{k}x_{k-1} \right\rbrace =FE\left\lbrace x_{k-1}x_{k-1}\right\rbrace +GE\left\lbrace \epsilon_{k}x_{k-1}\right\rbrace \]
		 
		 \begin{equation}\label{5.32}
		 \sigma^{2}e^{-\alpha}=F\sigma^{2}
		 \end{equation}
		 
		 假设 $ \varepsilon_{k} $ 是不相关的并且 $ E\left\lbrace \varepsilon_{k}\right\rbrace =0 $,则 $ E\left\lbrace \varepsilon_{k}\varepsilon_{k} \right\rbrace = 0 $。转换矩阵是 $ F=e^{-\alpha} $。接下来是由(5.31)定义的状态变量并取其期望值:
		 
		 \[ E\left\lbrace x_{k}^{2}\right\rbrace = F^{2}E\left\lbrace x_{k-1}x_{k-1}\right\rbrace+G^{2}E\left\lbrace \epsilon_{k}\epsilon{k}\right\rbrace \]
		 
		\begin{equation}\label{5.33}
		\sigma^{2}=\sigma^{2}F^{2}+G^{2}
		\end{equation}
		 
		  因为系统方差是  $ E\left\lbrace\varepsilon_{k}^{2}\right\rbrace=1 $ 。我们 将$ F=e^{-\alpha} $ 插入到5.33得到 $ G=\sigma\sqrt{1-e^{-2\alpha}}   $ 。完整的模型是
		  	
		  \[ x_{k}=e^{-\alpha}x_{k-1}+\sigma\sqrt{1-e^{-2\alpha}}\epsilon_{k} \]
		 
		 \[ b_{k}=x_{k} \]
		 
		 其中, $ E\left\lbrace \varepsilon_{k}\right\rbrace=0  $ , $E\left\lbrace \varepsilon_{k}\varepsilon_{j}\right\rbrace=\delta_{jk}$。
		 
		 理想情况下,随机过程应基于管理系统错误噪声的物理规律。 确切的表述通常是不可能的,因为底层物理学不是很好理解,或者因为实现理想的随机过程将产生麻烦的解决方案。 高斯马尔可夫模型(相关指数衰减)非常有用,只需要一个参数 $ \alpha $。