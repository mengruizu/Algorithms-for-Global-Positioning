\section[去相关和规范权]{去相关和规范权\\Decorrelation and Weight Nomalization}
用于求解最小二乘问题的大多数计算机程序通过行读取观测方程,并且将相关贡献存储在正规方程矩阵的匹配位置。从$A$的第$i$行的第$i$个观测方程可以立即存储在正规方程矩阵$A^{T}CA$中的正确位置
\begin{align}
(A^{T}CA)_{old} + \dfrac{1}{\sigma^{2}_{i}}A^{T}_{i}A_{i}\rightarrow (A^{T}CA)_{new}.
\end{align}
这个程序是正确的,只要个别观测是不相关的。如果它们是相关的,我们首先必须通过将$C$拆分成$W^{T}W$来解相关观测。然后
\begin{align}
A' = WA \quad \text{和} \quad (A')^{T}A' = A^{T}W^{T}WA = A^{T}CA.
\end{align}
这样的变换$W$不仅将变换后的观测值进行去相关,而且还可以保证这些方程具有权重1。

 由于协方差矩阵 $C^{-1}$是对称和正定的,我们可以使用$Cholesky$的方法进行因式分解:
 \begin{align}
 \Sigma _{b} = C^{-1} = W^{-1}C^{-T} \quad \text{或} \quad C = W^{T}W
 \end{align}
  因此描述的变换被给出
  \begin{align*}
  A' = WA \quad \text{和} \quad (\textbf{\textit{b}}')^{T} = W\textbf{\textit{b}} \quad \text{和} \quad  \textbf{\textit{e}}' = W\textbf{\textit{e}}.
  \end{align*}
  很容易发现,见下文,$\textbf{\textit{x}}$在这个特殊的转换下保持不变,并且变换观测的先验协方差矩阵被简化为   
  \begin{align}
  \Sigma_{b'} = I.
  \end{align}
协方差矩阵的分解和$A$和$b$的变换通常被称为观测值的去相关。在实践中,你只需要为每组相关观察值计算,然后求解方程组
\begin{align}
W^{-1}A' = A \quad \text{和} \quad  W^{-1}\textbf{\textit{b}}' =\textbf{\textit{b}}
\end{align}
对于$A'$和$\textbf{\textit{b}}'$通过回代计算。虽然$A$包含更多相关的行,$A'$的计算继续逐列。在分解之后,$A'$和$\textbf{\textit{b}}'$被添加到如(6.54)中所述的正规方程。

变换是$W$的列空间中的基础的变化。对于每组相关的观察值,唯一地确定该列空间。任何这样的线性变换使得$A^{T}CA$,$A^{T}Cb$,和$\textbf{\textit{x}}$保持不变。可以执行变换以同时确保相关观测方程的单位权重矩阵,如(6.57)中所示。

残差$\textbf{\textit{e}}^{T}C\textbf{\textit{e}}$ 的加权平方和等于变换残差的平方和$\textbf{\textit{e}}'=W\textbf{\textit{e}}$:
\begin{align*}
\textbf{\textit{e}}^{T}C\textbf{\textit{e}} = 
(W^{-1}\textbf{\textit{e}}')^{T}C(W^{-1}\textbf{\textit{e}}')=
\textbf{\textit{e}}'^{T}\textbf{\textit{e}}' =
\sum_{i=1}^m\textbf{\textit{e}}'^{2}_{i}.
\end{align*}
示例6.7 \ 我们给出一个详细说明该理论的数值示例,并使用协方差矩阵进行双差分相位观测。双差分观测由(10.16)定义。我们考虑具有$r = 3$个接收机的情况,其中一个是主接收机。主接收器对应于第一块列;第二和第三块列对应于两个漫游器接收器。所有接收机观察$s = 4$颗卫星。第一颗卫星作为参考卫星,我们在每个接收机有三个观测值。每个块矩阵是$s-1 \times s$:
\begin{align*}
D_{d} =
\begin{bmatrix}
-1 \quad 1 \quad 0 \quad 0 \quad 1 \quad-1 \quad 0 \quad 0 \quad 0 \quad 0 \quad 0 \quad 0 \\
-1 \quad 0 \quad 1 \quad 0 \quad 1 \quad 0 \quad-1 \quad 0 \quad 0 \quad 0 \quad 0 \quad 0 \\
-1 \quad 0 \quad 0 \quad 1 \quad 1 \quad 0 \quad 0 \quad-1 \quad 0 \quad 0 \quad 0 \quad 0 \\
-1 \quad 1 \quad 0 \quad 0 \quad 0 \quad 0 \quad 0 \quad 0 \quad 1 \quad-1 \quad 0 \quad 0 \\
-1 \quad 0 \quad 1 \quad 0 \quad 0 \quad 0 \quad 0 \quad 0 \quad 1 \quad 0 \quad-1 \quad 0 \\
-1 \quad 0 \quad 0 \quad 1 \quad 0 \quad 0 \quad 0 \quad 0 \quad 1 \quad 0 \quad 0 \quad-1 
\end{bmatrix}.
\end{align*}
协方差矩阵$ \Sigma_{d} = D_{d}(\sigma^{2}I)D^{T}_{d} = C^{-1}$是
\begin{align*}
\sigma^{2}
\begin{bmatrix}
4 & 2 & 2 & 2 & 1 & 1 \\
2 & 4 & 2 & 1 & 2 & 1 \\
2 & 2 & 4 & 1 & 1 & 2 \\
2 & 1 & 1 & 4 & 2 & 2 \\
1 & 2 & 1 & 2 & 4 & 2 \\
1 & 1 & 2 & 2 & 2 & 4 
\end{bmatrix}.
\end{align*}
第一步是将$C^{-1}$因子分解成$ W^{-1} (W^{-1} )^{T}$。结果是
\begin{align*}
W^{-1} = 
\begin{bmatrix}
2 & 0 & 0 & 0 & 0 & 0 \\
1 & \sqrt{3} & 0 & 0 & 0 & 0 \\
1 & \frac{1}{\sqrt{3}} & \frac{2\sqrt{2}}{\sqrt{3}} & 0 & 0 & 0 \\
1 & 0 & 0 & \sqrt{3} & 0 & 0 \\
\frac{1}{2} & \dfrac{\sqrt{3}}{2} & 0 & \dfrac{\sqrt{3}}{2} & \dfrac{3}{2} & 0 \\
\dfrac{1}{2} & \frac{1}{2\sqrt{3}} & \dfrac{\sqrt{2}}{\sqrt{3}} & \dfrac{\sqrt{3}}{2} & \dfrac{1}{2} & \sqrt{2} 
\end{bmatrix}.
\end{align*}
注意, $w^{-1}$ 只有正值,如$ \Sigma_{d}$。第二步是计算:
\begin{align*}
W = 
\begin{bmatrix}
\dfrac{1}{2} & 0 & 0 & 0 & 0 & 0 \\
-\dfrac{1}{2\sqrt{3}} & \dfrac{1}{\sqrt{3}} & 0 & 0 & 0 & 0 \\
-\dfrac{1}{2\sqrt{6}} & -\dfrac{1}{2\sqrt{6}} & \frac{\sqrt{3}}{2\sqrt{2}} & 0 & 0 & 0 \\
-\dfrac{1}{2\sqrt{3}} & 0 & 0 & \dfrac{1}{\sqrt{3}} & 0 & 0 \\
\frac{1}{6} & -\dfrac{1}{3} & 0 & -\dfrac{1}{3} & \dfrac{2}{3} & 0 \\
\dfrac{1}{6\sqrt{2}} & \dfrac{1}{6\sqrt{2}} & -\dfrac{\sqrt{2}}{\sqrt{2}} & -\dfrac{1}{3\sqrt{2}} & -\dfrac{1}{3\sqrt{2}} & \dfrac{1}{\sqrt{2}} 
\end{bmatrix}.
\end{align*}
注意,$W^{-1}$和$W $的对角项相互相反。零条目放置在两个矩阵中的相同位置,并且它们共享所有三角矩阵共有的其他属性:两个下(上)三角矩阵的乘积再次是较低(上)三角形,并且非奇异的,三角矩阵也是三角形。

在与原始相关观测方程的两侧的$W$进行左乘之后,独立观测可以像任何其他独立观测一样由最小二乘法程序读取。

最后,我们演示如何将各种类型的观察结合到一个最小二乘问题。关键是将单个观测方程除以其标准偏差。或者换句话说,将单个观测方程乘以相关权重的平方根。由此,观测方程变为无量纲;随后它们进入最小二乘问题。我们说观测方程可以归一化:
\begin{align}
WA\textbf{\textit{x}} = W\textbf{\textit{b}}-W\textbf{\textit{e}},
\end{align}
通过这个正则方程
\begin{align}
A^{T}W^{T}WA\textbf{\textit{x}}=A^{T}W^{T}W\textbf{\textit{b}}
\end{align}
或
\begin{align}
A^{T}CA\textbf{\textit{x}}=A^{T}C\textbf{\textit{b}}.
\end{align}
它们是具有权重矩阵 $ C=W^{T}W$的正确的方程式。

关于权重的最后一句话。任何观测方程可以乘以任何正常数;但你必须明白,这会改变权重,从而改变最小二乘问题的结果。